\documentclass[11pt]{article}
\usepackage[BoldFont,SlantFont,CJKchecksingle]{xeCJK}
\usepackage[top=0.5in,bottom=0.5in,left=1.25in,right=0.8in]{geometry}
\setCJKmainfont[BoldFont=SimHei]{SimSun}
\setCJKmonofont{SimSun}% 设置缺省中文字体
\parindent 0em   %段首缩进
%\usepackage{indentfirst}	%设置第一段也首行缩进
\linespread{1}	%设置行距
\addtolength{\parskip}{.4em}	%增加段间距0.4em
\usepackage{amsmath} % 插入数学公式
\usepackage[colorlinks,linkcolor=blue]{hyperref} 	%设置超链接

% 设置页眉页脚
\usepackage{fancyhdr}
\pagestyle{fancy}
\lhead{} 
\chead{} 
\rhead{\bfseries {ramayzhu0625@gmail.com}} 
\lfoot{} 
\cfoot{}
\rfoot{\thepage} 
\renewcommand{\headrulewidth}{0.4pt} 
\renewcommand{\footrulewidth}{0.4pt}

% \includegraphics[width = .8\textwidth]{pic.png}  图片的宽度会被缩放至页面宽度的百分之八十,图片的总高度会按比例缩放 

\begin{document}
	\title{Machine Learning Week-5}
	\author{ramay7}
	
	\maketitle % 显示标题
	\tableofcontents % 生成目录
	%\newpage
	
	\section{K-means Algorithm}
		Randomly initialize K cluster centroid $\mu_1, \mu_2\cdots, \mu_k \in R^{n}$.
		
		Repeate \{ \\
		.\qquad	for i = 1 to m\\
		.\qquad \qquad $C^{(i)}$ := index(from 1 to K) of cluster centroid closest to $X^{(i)}$ \\
		.\qquad for k = 1 to K\\
		.\qquad \qquad $\mu_k$ := average (mean) of points assiagned to cluster k\\
		\} \\
	\section{Principal Component Analysis Algorithm}
		PCA is not linear regression.
		\subsection{Data Preprocessing}
			Training set: $X^{(1)}, X^{(2)}, \cdots, X^{(m)}$ \\
			Prerocessing (feature scaling/mean normalization):\\
			.\qquad $\mu_j = \frac{1}{m}\sum_{i=1}^{m}X_{j}^{(i)}$ \\
			.\qquad Replace each $X_{j}^{(i)}$ with $X_j - \mu_j$ \\
			.\qquad If different features on different scales (e.g. $X_1$ = size of house, $X_2$ = number of bedrooms), scale features to have comparable range of values.
		\subsection{Reduce data from n-dimensions to K-dimensions}
			Compute "covariance matrix":\\
			.\qquad Sigma = $\frac{1}{m}\sum_{i=1}^{n}(x^{(i)})(x^{(i)})^{T}$\\
			Compute "eigenvectors" of matrix Sigma:
			.\qquad [U, S, V] = svd(Sigma)\\
			where svd is the abbreviation of singular value decomposition.\\
			.\qquad $ U_{reduce} = U(:, 1:K) $;
		\subsection{Reconstruction from compressed representation}
			.\qquad $X_{approx}^{(i)} = U_{reduce}Z^{(i)}$ \\
			where $U_{reduce}$ is a n by k matrix and $Z^{(i)}$ is a k by 1 matrix.
	\section{Choosing K(number of principal conponents)}
		Average squared projection error:
		$$
		\frac{1}{m}\sum_{i=1}^{m}|| x^{(i)} - x^{(i)}_{approx}|| ^ 2
		$$
		Total variation in the data:
		$$
		\frac{1}{m}\sum_{i=1}^{m}|| x^{(i)} || ^ 2
		$$
		Typically, choose k to be smallest value so that:
		$$
		A = \frac{\frac{1}{m}\sum_{i=1}^{m}|| x^{(i)} - x^{(i)}_{approx} ||^2}{\frac{1}{m}\sum_{i=1}^{m}|| x^{(i)} ||^ 2} \leq 0.01
		$$
		which indicates 99\% of variance is retained". \\
		
		For $[U, S, V] = svd(Sigma)$:
		$$
		\begin{aligned}
		S = 
		\left[
		\begin{matrix}
			S_{11} & 0 & 0 & ... & 0\\
			0 & S_{22} & 0 & ... & 0\\
			0 & 0 & S_{33} & ... & 0\\
			. & . & . & ... & 0 \\
			0 & 0 & 0 & ... & S_{nn}  
		\end{matrix}
		\right] \\
		\end{aligned}
		$$
		For given k:
		$$
		A = 1 - \frac{\sum_{i=1}^{k}S_{ii}}{\sum_{i=1}^{n}S_{ii}}
		$$
	\section{Application of PCA}
	\begin{itemize}
		\item To compress the data so it takes up less computer memory/disk space
		\item To reduce the dimension of the input data so as to speed up a learning algorithm
		\item To visualize high-dimensional data (by choosing k = 2 or k = 3)
	\end{itemize}
\end{document}
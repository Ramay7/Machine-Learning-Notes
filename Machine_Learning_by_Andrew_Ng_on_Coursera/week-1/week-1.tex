\documentclass[11pt]{article}
\usepackage[BoldFont,SlantFont,CJKchecksingle]{xeCJK}
\usepackage[top=0.5in,bottom=0.5in,left=1.25in,right=0.8in]{geometry}
\setCJKmainfont[BoldFont=SimHei]{SimSun}
\setCJKmonofont{SimSun}% 设置缺省中文字体
\parindent 2em   %段首缩进
\usepackage{indentfirst}	%设置第一段也首行缩进
\linespread{1}	%设置行距
\addtolength{\parskip}{.4em}	%增加段间距0.4em
\usepackage{amsmath} % 插入数学公式
\usepackage[colorlinks,linkcolor=blue]{hyperref} 	%设置超链接

% 设置页眉页脚
\usepackage{fancyhdr}
\pagestyle{fancy}
\lhead{} 
\chead{} 
\rhead{\bfseries {ramayzhu0625@gmail.com}} 
\lfoot{} 
\cfoot{}
\rfoot{\thepage} 
\renewcommand{\headrulewidth}{0.4pt} 
\renewcommand{\footrulewidth}{0.4pt}

% \includegraphics[width = .8\textwidth]{pic.png}  图片的宽度会被缩放至页面宽度的百分之八十,图片的总高度会按比例缩放 

\begin{document}
	\title{Machine Learning Week-1}
	\author{ramay7}
	
	\maketitle % 显示标题
	\tableofcontents % 生成目录
	%\newpage
	
	\section{Introduction}
		
		The popular definition of Machine Learning is :" A computer program is said to learn from experience E with respect to some class of tasks T and performance measure P, if \textbf{its performance at tasks in T, as measured by P, improves with Experience E.}"

		In general, any machine learning problem can be assigned to one of two broad classification: \textbf{Supervised learning and Unsupervised learning}.

		In supervised learning, we are given a data set and \textbf{already} know what our correct output should look like, having the idea that there is a relationship between the input and the output. Supervised learning problems are categorized into \textbf{"regression" and "classification" problems}. In a regression problem, we are trying to map input variables to some \textbf{continuous function}. In a classification problem, we are instead trying to predict results in a \textbf{discrete output}.

		Unsupervised learning allows us to approach problems with \textbf{little or no idea} what our results should look like. We can derive structure frome data where we don't necessarily know the effect of the varaibles. We can derive this structure by clustering the data \textbf{based on relationships among the variables} in the data. With unsupervised learning there is no feedback based on the prediction results.

	\section{Model and Cost Function}

		To describe the supervised learning problem slightly more formally, our goal is, given a training set, to learn a function h: X $\rightarrow$ Y so that h(x) is a "good" predictor for the corresponding value of y. For historical reasons, this function h is called a \textbf{hypothesis}.

		Specificly, we can get a linear regression hypothesis: 

		$$
		h(x) = \theta _0 + \theta _1 x
		$$

		We can measure the accuracy of our hypothesis function by using a \textbf{cost funciton}. This takes an average difference (actually a fancier version of an average) of all the results of the hypothesis with inputs from x's and the actual output y's.

		$$
		J(\theta _{0},\theta _{1})=\frac{1}{2m} \sum_{i=1}^{m}(h_{\theta}(x_i)-y_i)^2
		$$

		This function is otherwise called the "Squared error function", or "Mean squared error".

	\section{Parameter Learning} 

		Now we need to estimate the parameters in the hypothesis function. That's where \textbf{gradient descent} comes in.

		We put $\theta _0$ on the x axis and $\theta _1$ on the y axis, with the cost function on the vertical z axis. The points on our graph will be the result of the cost function using our hypothesis with those specific theta parameters. The graph below depicts such a setup. 

		\includegraphics[width = .8\textwidth]{gradient-descent.png}

		We will know that we have succeeded when our cost function is at the very bottom of the pits in our graph. I.e. when its value is the minimum. The red arrows show the minimum points in the graph.

		The way we do this by taking \textbf{the derivative (the tangential line to a function) of our cost function}. The slope of the tangent is the derivative at that point and it will give us a direction to move towards. We make steps down the cost function in the direction with the steepest descent. The size of each step is determined by the parameter $\alpha$, which is called the \textbf{learning rate}.

		The gradient descent algorithm is:

		\textbf{repeat until convergence:\{ 
				\begin{center}
					$\theta _{j}:=\theta _{j}-\alpha \frac{\partial}{\partial \theta _{j}}J(\theta _{0}, \theta _{1})$ \\
					where j=0,1 represents the feature index number
				\end{center}
			\qquad \} }

		At each iteration j, one should \textbf{simultaneously} update the paramerers $\theta _1, \theta _2, \cdots ,\theta _n$. That means updating all parameters together after calculating all parameters at one iteration.

		If $\alpha$ is too small, gradient descent
		can be too slow. And on the other hand, if $\alpha$ is too large, gradient descent can overshoot the minimum, it may fail to converge, or even diverge.

	\subsection{Gradient Descent For Linear Regression}

		We can substitute our actual cost function and our actual hypothesis function and modify the equation to:

		\textbf{repeat until convergence: \{	
			\begin{align}
				\theta _{0} &:= \theta _0 - \alpha \frac{1}{m}\sum_{i=1}^{m}(h_{\theta}(x_i)-y_i) \\
				\theta _{1} &:= \theta _1 - \alpha \frac{1}{m}\sum_{i=1}^{m}((h_{\theta}(x_i)-y_i)x_i)
			\end{align}
			\qquad \} }

		where m is the size of the training set, $\theta _0$ a constant that will be changing simultaneously with $\theta _1$ and $x_i, y_i$ are values of the given training set(data).

		We can easily prove these two equations with the definition of linear regression hypothesis and gradient descent.

		Note that, while gradient descent can be susceptible to local minima in general, the optimization problem we have posed here for linear gregression has only one global, and no other local,  optima, thus gradient descent always converges (assuming the learning rate $\alpha$ is not too large) to the global minimum. Indeed, J is a convex quadratic function. 

		In other words, there are some problems who may have local optima. And evenly, we may at the worse condition (the top point on the contour plots) at first. So we need to test our $\theta _0, \theta _1$ and $\alpha$ many times.

	\section{Linear Algebra Review}

		For a matrix B with m rows and o colums multiplied by a matrix A with n rows and m columns, we can get a matrix with n rows and o columns.

		In a general way, for two matrices A and B, $A\times B \neq B \times A$. However, $A \times E = E \times A$, where A is a square matrix and E is a unit matrix.

		It is also necessary to know the concept of Inverse and Transpose.

\end{document}